\section{Distribución en el tiempo }
Los proyectos ABP que se proponen están diseñados para integrarse perfectamente en un entorno educativo \textbf{humanista-constructivista}, apuntando a desarrollar \textbf{competencias integrales} en las y los  estudiantes a través de la exploración autónoma y la experiencia directa. Al centrarse en el \textbf{aprendizaje activo-olaborativo}, estos proyectos no solo mejora las habilidades académicas tradicionales, sino que también fomenta \textbf{habilidades socioemocionales} y valores esenciales como la empatía, la responsabilidad y la creatividad.
\\  \\
La \textbf{duración de 9 meses} permite una exploración profunda de los temas, dando tiempo suficiente para que los estudiantes se involucren significativamente con el material, realicen investigaciones exhaustivas y desarrollen los proyectos con \textbf{impacto real y tangible}. Al apoyarse en múltiples \textbf{pilares educativos}, estos ABPs busca proporcionar una experiencia de aprendizaje que prepare a los estudiantes no solo para exámenes y evaluaciones, sino para los desafíos de la vida real.
\\ \\
Los proyectos propuestos están diferenciados por grado académico para alinearse con las etapas de desarrollo de los estudiantes y sus intereses específicos.
\begin{itemize}
    \item Para los estudiantes de \textbf{Primer Grado}, el enfoque está en las \textbf{Artes}, permitiéndoles explorar diversas formas de expresión y creatividad.
    \item El \textbf{Segundo Grado} se centra en las \textbf{Ciencias}, donde los estudiantes pueden investigar y aplicar conceptos científicos a través de experimentos prácticos.
    \item Por último, el \textbf{Tercer Grado} se orienta hacia el \textbf{Emprendimiento}, ofreciendo a los estudiantes la oportunidad de diseñar y lanzar sus propias iniciativas, aplicando habilidades de liderazgo y gestión.
\end{itemize}
La distribución temporal para realizar los ABPs en todos los grados, es la siguiente:
\begin{enumerate}
    \item \textbf{Justificación.}
    \begin{itemize}
        \item [Sem. 0-0:] Presentación del proyecto. \\
        \begin{tikzpicture}
             \calendar(mycal)[
             dates=2024-08-1 to 2024-08-31,
             week list,
             month label above centered,
             day headings=black, % Nombres de la sem.
             day letter headings, % Nombres de la sem. 
             month text=\textbf{\textcolor{black}{\%mt \%y0}},
             if={(Sunday) [black!30]},
             if={(Saturday) [black!30]}
             %if={(Saturday) [black!50]},
             %if={(day of month=26) [blue]},
             %if={(day of month=27) [blue]},
             %if={(day of month=28) [blue]},
             %if={(day of month=29) [blue]},
             %if={(day of month=30) [blue]},
             ];
             \node[anchor=east] at ([xshift=-0.5em]mycal-2024-08-26.west){\textbf{0:}};
        \end{tikzpicture}

    \end{itemize}

    \item \textbf{Inicio de Proyecto.}
    \begin{itemize}
        \item [Sem. 1-2:] Presentación del Desafío o Problema Central.
        \item [Sem. 3-4:] Definición de Objetivos de Aprendizaje.\\

            \begin{tikzpicture}
             \calendar(mycal)[
             dates=2024-09-1 to 2024-09-30,
             week list,
             month label above centered,
             day headings=black, % Nombres de la sem.
             day letter headings, % Nombres de la sem. 
             month text=\textbf{\textcolor{black}{\%mt \%y0}},
             if={(Sunday) [black!30]},
             if={(Saturday) [black!30]},
             if={(day of month=16) [black!50]},
             if={(day of month=27) [black!50]},
             ];
             \node[anchor=east] at ([xshift=-1em]mycal-2024-09-02.west) {\textbf{1:}};
            \node[anchor=east] at ([xshift=-1em]mycal-2024-09-09.west) {\textbf{2:}};
             \node[anchor=east] at ([xshift=-0.5em]mycal-2024-09-16.west) {\textbf{3:}};
             \node[anchor=east] at ([xshift=-0.5em]mycal-2024-09-23.west) {\textbf{4:}};
             \node[anchor=east] at ([xshift=-0.5em]mycal-2024-09-30.west) 
             {5:};
        \end{tikzpicture}
        
    \end{itemize}
    
    \item \textbf{Planificación y Diseño.}
        \begin{itemize}
        \item [Sem. 5-6:] Formulación de Preguntas Guía
        \item [Sem. 7-8:] Selección de Recursos 

            \begin{tikzpicture}
             \calendar(mycal)[
             dates=2024-10-1 to 2024-10-31,
             week list,
             month label above centered,
             day headings=black, % Nombres de la sem.
             day letter headings, % Nombres de la sem. 
             month text=\textbf{\textcolor{black}{\%mt \%y0}},
             if={(Sunday) [black!30]},
             if={(Saturday) [black!30]},
             if={(day of month=25) [black!50]},
             ];
             \node[anchor=east] at ([xshift=-2.5em]mycal-2024-10-01.west) {\textbf{5:}};
             \node[anchor=east] at ([xshift=-1em]mycal-2024-10-07.west) {\textbf{6:}};
            \node[anchor=east] at ([xshift=-0.5em]mycal-2024-10-14.west) {\textbf{7:}};
             \node[anchor=east] at ([xshift=-0.5em]mycal-2024-10-21.west) {\textbf{8:}};
             \node[anchor=east] at ([xshift=-0.5em]mycal-2024-10-28.west)
             {9:};
        \end{tikzpicture}
        
    \end{itemize}
    
    \item \textbf{Investigación y Desarrollo.}
        \begin{itemize}
        \item [Sem. 9-16:] Investigación Activa
        \item [Sem. 17-24:] Desarrollo del Proyecto \\

            \begin{tikzpicture}
             \calendar(mycal)[
             dates=2024-11-01 to 2024-11-30,
             week list,
             month label above centered,
             day headings=black, % Nombres de la sem.
             day letter headings, % Nombres de la sem.
             month text=\textbf{\textcolor{black}{\%mt \%y0}},
             if={(Sunday) [black!30]},
             if={(Saturday) [black!30]},
             if={(day of month=18) [black!50]},
             if={(day of month=29) [black!50]}
             ];
             \node[anchor=east] at ([xshift=-7.5em]mycal-2024-11-01.west) {\textbf{9:}};
             \node[anchor=east] at ([xshift=-1em]mycal-2024-11-04.west) {\textbf{10:}};
             \node[anchor=east] at ([xshift=-0.5em]mycal-2024-11-11.west) {\textbf{11:}};
            \node[anchor=east] at ([xshift=-0.5em]mycal-2024-11-18.west) {\textbf{12:}};
             \node[anchor=east] at ([xshift=-0.5em]mycal-2024-11-25.west) {\textbf{13:}};
        \end{tikzpicture}
        \\ \\
        
            \begin{tikzpicture}
             \calendar(mycal)[
             dates=2024-12-01 to 2024-12-31,
             week list,
             month label above centered,
             day headings=black, % Nombres de la sem.
             day letter headings, % Nombres de la sem.
             month text=\textbf{\textcolor{black}{\%mt \%y0}},
             if={(Sunday) [black!30]},
             if={(Saturday) [black!30]},
             if={(day of month=19) [black!50]},
             if={(day of month=20) [black!50]},
             if={(day of month=23) [black!50]},
             if={(day of month=24) [black!50]},
             if={(day of month=25) [black!50]},
             if={(day of month=26) [black!50]},
             if={(day of month=27) [black!50]},
             if={(day of month=30) [black!50]},
             if={(day of month=31) [black!50]}
             ];
             \node[anchor=east] at ([xshift=-10em]mycal-2024-12-01.west) {\textbf{14:}};
             \node[anchor=east] at ([xshift=-1em]mycal-2024-12-02.west) {\textbf{15:}};
             \node[anchor=east] at ([xshift=-1em]mycal-2024-12-09.west) {\textbf{16:}};
             \node[anchor=east] at ([xshift=-0.5em]mycal-2024-12-16.west) {\textbf{16} \textit{bis:}};
        \end{tikzpicture}
             \\ \\ 
             \begin{tikzpicture}
             \calendar(mycal)[
             dates=2025-01-01 to 2025-01-31,
             week list,
             month label above centered,
             day headings=black, % Nombres de la sem.
             day letter headings, % Nombres de la sem.
             month text=\textbf{\textcolor{black}{\%mt \%y0}},
             if={(Sunday) [black!30]},
             if={(Saturday) [black!30]},
             if={(day of month=1) [black!50]},
             if={(day of month=2) [black!50]},
             if={(day of month=3) [black!50]},
             if={(day of month=31) [black!50]}
             ];
             \node[anchor=east] at ([xshift=-1.5em]mycal-2025-01-06.west) {\textbf{16} \textit{ter:}};
             \node[anchor=east] at ([xshift=-1em]mycal-2025-01-13.west) {\textbf{17:}};
             \node[anchor=east] at ([xshift=-1em]mycal-2025-01-20.west) {\textbf{18:}};
             \node[anchor=east] at ([xshift=-1em]mycal-2025-01-27.west) {\textbf{19:}};
         \end{tikzpicture}
         \\ \\
         
         \begin{tikzpicture}
             \calendar(mycal)[
             dates=2025-02-01 to 2025-02-28,
             week list,
             month label above centered,
             day headings=black, % Nombres de la sem.
             day letter headings, % Nombres de la sem.
             month text=\textbf{\textcolor{black}{\%mt \%y0}},
             if={(Sunday) [black!30]},
             if={(Saturday) [black!30]},
             if={(day of month=3) [black!50]},
             if={(day of month=28) [black!50]}
             ];
             \node[anchor=east] at ([xshift=-1.5em]mycal-2025-02-03.west) {\textbf{20:}};
             \node[anchor=east] at ([xshift=-1em]mycal-2025-02-10.west) {\textbf{21:}};
             \node[anchor=east] at ([xshift=-1em]mycal-2025-02-17.west) {\textbf{22:}};
             \node[anchor=east] at ([xshift=-1em]mycal-2025-02-24.west) {\textbf{23:}};
         \end{tikzpicture}
         \\ \\

         \begin{tikzpicture}
             \calendar(mycal)[
             dates=2025-03-01 to 2025-03-31,
             week list,
             month label above centered,
             day headings=black, % Nombres de la sem.
             day letter headings, % Nombres de la sem.
             month text=\textbf{\textcolor{black}{\%mt \%y0}},
             if={(Sunday) [black!30]},
             if={(Saturday) [black!30]},
             if={(day of month=17) [black!50]},
             if={(day of month=28) [black!50]}
             ];
             \node[anchor=east] at ([xshift=-1.5em]mycal-2025-03-03.west) {\textbf{24:}};
             \node[anchor=east] at ([xshift=-1em]mycal-2025-03-10.west)
             {25:};
             \node[anchor=east] at ([xshift=-1em]mycal-2025-03-17.west)
             {26:};
             \node[anchor=east] at ([xshift=-1em]mycal-2025-03-24.west)
             {27:};
             \node[anchor=east] at ([xshift=-1em]mycal-2025-03-31.west)
            {28:};
         \end{tikzpicture}
          
         
    \end{itemize}
    \item \textbf{Creación y Colaboración.}
        \begin{itemize}
        \item [Sem. 25-26:]  Trabajo en Equipo
        \item [Sem. 27-28:]  Uso de Tecnología
        \\

\begin{tikzpicture}
             \calendar(mycal)[
             dates=2025-03-01 to 2025-03-31,
             week list,
             month label above centered,
             day headings=black, % Nombres de la sem.
             day letter headings, % Nombres de la sem.
             month text=\textbf{\textcolor{black}{\%mt \%y0}},
             if={(Sunday) [black!30]},
             if={(Saturday) [black!30]},
             if={(day of month=17) [black!50]},
             if={(day of month=28) [black!50]}
             ];
             \node[anchor=east] at ([xshift=-1.5em]mycal-2025-03-03.west)
            {24:};
             \node[anchor=east] at ([xshift=-1em]mycal-2025-03-10.west)
             {\textbf{25:}};
             \node[anchor=east] at ([xshift=-1em]mycal-2025-03-17.west)
             {\textbf{26:}};
             \node[anchor=east] at ([xshift=-1em]mycal-2025-03-24.west)
             {\textbf{27:}};
             \node[anchor=east] at ([xshift=-1em]mycal-2025-03-31.west)
            {\textbf{28:}};
         \end{tikzpicture}
         \\ \\ 
         \begin{tikzpicture}
             \calendar(mycal)[
             dates=2025-04-01 to 2025-04-30,
             week list,
             month label above centered,
             day headings=black, % Nombres de la sem.
             day letter headings, % Nombres de la sem.
             month text=\textbf{\textcolor{black}{\%mt \%y0}},
             if={(Sunday) [black!30]},
             if={(Saturday) [black!30]},
             if={(day of month=14) [black!50]},
             if={(day of month=15) [black!50]},
             if={(day of month=16) [black!50]},
             if={(day of month=17) [black!50]},
             if={(day of month=18) [black!50]},
             if={(day of month=21) [black!50]},
             if={(day of month=22) [black!50]},
             if={(day of month=23) [black!50]},
             if={(day of month=24) [black!50]},
             if={(day of month=25) [black!50]},
             if={(day of month=30) [black!50]}
             ];
             \node[anchor=east] at ([xshift=-2.5em]mycal-2025-04-01.west)
             {\textbf{28:}};
             \node[anchor=east] at ([xshift=-1em]mycal-2025-04-07.west)
             {29:};
             \node[anchor=east] at ([xshift=-0.5em]mycal-2025-04-28.west)
             {30:};
              \end{tikzpicture}
        
    \end{itemize}
    
    \item \textbf{Presentación y Evaluación.}
        \begin{itemize}
        \item [Sem. 29-30:] Presentación del Proyecto
        \item [Sem. 31-32:] Reflexión y Evaluación
        \\ \\

        \begin{tikzpicture}
             \calendar(mycal)[
             dates=2025-04-01 to 2025-04-30,
             week list,
             month label above centered,
             day headings=black, % Nombres de la sem.
             day letter headings, % Nombres de la sem.
             month text=\textbf{\textcolor{black}{\%mt \%y0}},
             if={(Sunday) [black!30]},
             if={(Saturday) [black!30]},
             if={(day of month=14) [black!50]},
             if={(day of month=15) [black!50]},
             if={(day of month=16) [black!50]},
             if={(day of month=17) [black!50]},
             if={(day of month=18) [black!50]},
             if={(day of month=21) [black!50]},
             if={(day of month=22) [black!50]},
             if={(day of month=23) [black!50]},
             if={(day of month=24) [black!50]},
             if={(day of month=25) [black!50]},
             if={(day of month=30) [black!50]}
             ];
             \node[anchor=east] at ([xshift=-2.5em]mycal-2025-04-01.west)
             {28:};
             \node[anchor=east] at ([xshift=-1em]mycal-2025-04-07.west)
             {\textbf{29:}};
             \node[anchor=east] at ([xshift=-0.5em]mycal-2025-04-28.west)
             {\textbf{30:}};
              \end{tikzpicture}
             \\ \\ 
        
        \begin{tikzpicture}
             \calendar(mycal)[
             dates=2025-05-01 to 2025-05-31,
             week list,
             month label above centered,
             day headings=black, % Nombres de la sem.
             day letter headings, % Nombres de la sem.
             month text=\textbf{\textcolor{black}{\%mt \%y0}},
             if={(Sunday) [black!30]},
             if={(Saturday) [black!30]},
             if={(day of month=1) [black!50]},
             if={(day of month=5) [black!50]},
             if={(day of month=15) [black!50]},
             if={(day of month=30) [black!50]}
             ];
             \node[anchor=east] at ([xshift=-5.5em]mycal-2025-05-01.west)
             {\textbf{30:}};
             \node[anchor=east] at ([xshift=-1em]mycal-2025-05-05.west)
             {\textbf{31:}};
             \node[anchor=east] at ([xshift=-0.5em]mycal-2025-05-12.west)
             {\textbf{32:}};
             \node[anchor=east] at ([xshift=-0.5em]mycal-2025-05-19.west)
             {33:};
             \node[anchor=east] at ([xshift=-0.5em]mycal-2025-05-26.west)
             {34:};
              \end{tikzpicture}
    \end{itemize}
    
    \item \textbf{Reflezión y Retroalimentación.}
        \begin{itemize}
        \item [Sem. 33-34:] Retroalimentación Continua
        \item [Sem. 35-36:] Reflexión Final
        \\

                \begin{tikzpicture}
             \calendar(mycal)[
             dates=2025-05-01 to 2025-05-31,
             week list,
             month label above centered,
             day headings=black, % Nombres de la sem.
             day letter headings, % Nombres de la sem.
             month text=\textbf{\textcolor{black}{\%mt \%y0}},
             if={(Sunday) [black!30]},
             if={(Saturday) [black!30]},
             if={(day of month=1) [black!50]},
             if={(day of month=5) [black!50]},
             if={(day of month=15) [black!50]},
             if={(day of month=30) [black!50]}
             ];
             \node[anchor=east] at ([xshift=-5.5em]mycal-2025-05-01.west)
             {30:};
             \node[anchor=east] at ([xshift=-1em]mycal-2025-05-05.west)
             {31:};
             \node[anchor=east] at ([xshift=-0.5em]mycal-2025-05-12.west)
             {32:};
             \node[anchor=east] at ([xshift=-0.5em]mycal-2025-05-19.west)
             {\textbf{33:}};
             \node[anchor=east] at ([xshift=-0.5em]mycal-2025-05-26.west)
             {\textbf{34:}};
              \end{tikzpicture}
              \\ \\
              \begin{tikzpicture}
             \calendar(mycal)[
             dates=2025-06-01 to 2025-06-30,
             week list,
             month label above centered,
             day headings=black, % Nombres de la sem.
             day letter headings, % Nombres de la sem.
             month text=\textbf{\textcolor{black}{\%mt \%y0}},
             if={(Sunday) [black!30]},
             if={(Saturday) [black!30]},
             if={(day of month=27) [black!50]}
             ];
             \node[anchor=east] at ([xshift=-1em]mycal-2025-06-02.west)
             {\textbf{35:}};
             \node[anchor=east] at ([xshift=-1em]mycal-2025-06-09.west)
             {\textbf{36:}};
              \end{tikzpicture}
    \end{itemize}
    
\end{enumerate}