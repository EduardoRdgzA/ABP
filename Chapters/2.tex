\section{Justificación}
En el tercer grado de secundaria, los proyectos orientados al emprendimiento incorporan \textbf{Aprendizaje personalizado}, lo que permite a cada estudiante explorar y desarrollar sus propias ideas de negocio dentro de un marco de \textbf{Aprendizaje activo y para la vida}. Estos proyectos se enriquecen con la \textbf{Tecnología educativa} y la \textbf{Investigación}, elementos que estimulan la \textbf{Innovación y Creatividad} al resolver problemas reales y ofrecer soluciones viables. La integración de la \textbf{Neuroeducación} facilita un aprendizaje adaptado a las capacidades cognitivas de los alumnos, mientras que la promoción de los \textbf{Valores y self-System} asegura un desarrollo ético y responsable, preparando a los estudiantes para actuar con \textbf{Humanismo, Acción y Servicio}.
\\ \\ 
El \textbf{Perfil de Egreso} de estos estudiantes destaca por su capacidad en \textbf{Razonamiento lógico y matemático} y su habilidad para \textbf{Aprender y Construir con Tecnología}. Los proyectos de emprendimiento fomentan el \textbf{Trabajo en Equipo y Comunicación}, competencias cruciales para el éxito en cualquier entorno profesional y personal. Además, la \textbf{Proyección Internacional}y la capacidad trilingüe preparan a los estudiantes para operar en un mercado global, potenciando su \textbf{Crecimiento Personal} y la \textbf{Autogestión del Aprendizaje para la Vida}. En este contexto, los alumnos no solo se desarrollan como emprendedores sino también como ciudadanos globales, capaces de contribuir significativamente a la sociedad.
\\ \\ 
\textbf{Motivos Suficientes:} \\
\begin{itemize}
    \item \textit{Desarrollo de habilidades empresariales:}
    A través de proyectos de emprendimiento, los estudiantes aprenden sobre gestión de proyectos, financiación, marketing y otros aspectos cruciales para iniciar y administrar negocios.
    \item \textit{Estímulo de la autonomía y la responsabilidad: }
    Gestionar un proyecto desde la idea hasta la ejecución fomenta la autonomía y la responsabilidad personal.
\end{itemize}
\textbf{Motivos Necesarios:}
\begin{itemize}
    \item \textit{Preparación para el futuro laboral: }
     Entender los principios del emprendimiento es crucial en una economía globalizada y competitiva.
     \item \textit{Incentivo a la innovación social: }
    Proyectos enfocados en resolver problemas sociales cultivan la conciencia social y pueden inspirar a los estudiantes a aplicar soluciones innovadoras en el futuro.
\end{itemize}
\newpage
\section{Inicio del Proyecto}
\begin{itemize}
  \item \textbf{Semana 1-2: Presentación del Desafío o Problema Central}
  \begin{itemize}
    \item Introducción a un desafío científico relevante, como un problema ambiental, médico o tecnológico, que requiera una solución basada en la investigación científica.
    \item Sesiones de brainstorming para identificar los aspectos clave del problema y plantear hipótesis preliminares.
  \end{itemize}
  \item \textbf{Semana 3-4: Definición de Objetivos de Aprendizaje}
  \begin{itemize}
    \item Identificación de los conceptos científicos clave y las habilidades de investigación que los estudiantes necesitan desarrollar.
    \item Elaboración de un plan de estudio que incluya metas específicas de aprendizaje y competencias científicas.
  \end{itemize}
\end{itemize}

{\large \textbf{Recursos}}\\
\textit{Condiciones Iniciales:} Requirimientos para los alumnos puedan llevar a cabo el Aprendizaje Basado en Proyectos (ABP):
\begin{itemize}
    \item \textbf{Habilidades cognitivas:}
    \begin{itemize}
        \item \textbf{Pensamiento crítico:}\\
        Los alumnos deben ser capaces de analizar información, identificar problemas, formular hipótesis y evaluar soluciones.
        \item[\textit{e. gr.}]Evaluación de la validez de diferentes estudios científicos sobre el cambio climático. Análisis de datos experimentales para determinar la eficacia de diferentes tipos de fertilizantes en el crecimiento de plantas.
        \item \textbf{Creatividad: }\\
        Deben ser capaces de generar ideas originales y pensar de manera innovadora para encontrar soluciones a los problemas.
        \item[\textit{e. gr.}]Diseño de un experimento para probar nuevas formas de purificación de agua usando materiales reciclables. Innovación en la creación de un modelo de ecosistema que simula interacciones entre especies.
        \item \textbf{Resolución de problemas:}\\
        Deben ser capaces de aplicar sus conocimientos y habilidades para resolver problemas de manera efectiva.
        \item[\textit{e. gr.}]Investigación y desarrollo de soluciones para combatir la resistencia a los antibióticos. Implementación de métodos para mejorar la eficiencia energética en la escuela.
        \item \textbf{Toma de decisiones:} \\
        Deben ser capaces de evaluar diferentes opciones y tomar decisiones informadas.
        \item[\textit{e. gr.}]Selección de la técnica más adecuada para analizar muestras geológicas. Decisión sobre el mejor enfoque para un proyecto de conservación de especies en peligro de extinción.
        \item \textbf{Metacognición:}\\
        Deben ser capaces de reflexionar sobre su propio proceso de aprendizaje y identificar sus fortalezas y debilidades.
         \item[\textit{e. gr.}]
        Reflexión sobre los propios métodos de investigación científica y ajuste de hipótesis basadas en resultados experimentales. Autoevaluación del progreso en el dominio de técnicas avanzadas de laboratorio.
    \end{itemize}
    \item \textbf{Habilidades sociales:}
    \begin{itemize}
        \item \textbf{Trabajo en equipo: }\\
        Los alumnos deben ser capaces de trabajar de manera colaborativa con otros para alcanzar un objetivo común.
        \item[\textit{e. gr.}] Colaboración en proyectos de investigación para estudiar el impacto de los pesticidas en la biodiversidad local. Coordinación de un grupo para construir un prototipo de un vehículo solar.
        \item \textbf{Comunicación: }\\
        Deben ser capaces de comunicar sus ideas de manera efectiva, tanto oralmente como por escrito.
        \item[\textit{e. gr.}]Presentación de un proyecto de ciencias en una feria científica escolar. Redacción de un artículo para un blog de ciencia joven sobre los últimos avances en biotecnología.
        \item \textbf{Liderazgo:}\\
        Deben ser capaces de asumir roles de liderazgo y motivar a otros a trabajar en equipo.
        \item[\textit{e. gr.}]Liderar un equipo en la competencia de robótica, asegurando que todas las partes del proyecto estén integradas. Supervisión de un proyecto de reforestación escolar, incluyendo la organización de tareas y la distribución de recursos.
        \item \textbf{Empatía:}\\
         Deben ser capaces de comprender y respetar los puntos de vista de los demás.
         \item[\textit{e. gr.}] Proyectos de ciencias que incluyen el desarrollo de tecnologías asistivas para personas con discapacidades. Estudios sobre el impacto ambiental de la actividad humana y su efecto en comunidades locales.
         \item \textbf{Responsabilidad: }\\
         Deben ser responsables de su propio aprendizaje y del trabajo que realizan en equipo.
         \item[\textit{e. gr.}]
         Responsabilidad en el mantenimiento y cuidado de un laboratorio de ciencias escolar. Compromiso con el seguimiento a largo plazo de un estudio de calidad del aire en la comunidad.
    \end{itemize}
    \item \textbf{Habilidades técnicas:}
    \begin{itemize}
        \item \textbf{Investigación:}\\
        Los alumnos deben ser capaces de investigar información de manera efectiva utilizando diferentes fuentes.
        \item[\textit{e. gr.}]Búsqueda de información sobre los últimos avances en energía renovable. Estudio profundo sobre la historia y evolución de teorías astronómicas.
        \item \textbf{Manejo de información:}\\
         Deben ser capaces de organizar y analizar información de manera eficiente.
         \item[\textit{e. gr.}]Organización de datos obtenidos de experimentos en ecología para análisis estadístico. Sistematización de observaciones en estudios de campo sobre biodiversidad.
         \item \textbf{Uso de tecnología:}\\
         Deben ser capaces de utilizar las herramientas tecnológicas de manera adecuada para apoyar su aprendizaje.
         \item[\textit{e. gr.}] Organización de datos obtenidos de experimentos en ecología para análisis estadístico. Sistematización de observaciones en estudios de campo sobre biodiversidad.
         \item \textbf{Gestión del tiempo:}\\
         Deben ser capaces de planificar y gestionar su tiempo de manera efectiva para completar las tareas del proyecto.
         \item[\textit{e. gr.}]Planificación de las fases de un proyecto científico para observar el ciclo de vida de una especie de mariposa. Administración del tiempo en un proyecto de investigación durante el año escolar, asegurando la recopilación de datos en las diferentes estaciones.

    \end{itemize}
    \item \textbf{Actitudes:}
    \begin{itemize}
        \item \textbf{Motivación:}\\
        Los alumnos deben estar motivados para aprender y participar activamente en el proyecto.
        \item[\textit{e. gr.}]Participación en concursos nacionales de ciencia e ingeniería como motivación para desarrollar proyectos innovadores. Establecimiento de metas personales para aprender nuevas técnicas de análisis científico.
        \item \textbf{Interés:}\\
        Deben estar interesados en el tema del proyecto.
        \item[\textit{e. gr.}] Involucramiento en clubes de ciencia y tecnología para explorar intereses en diferentes campos como la física cuántica o la biología molecular. Asistencia a seminarios y conferencias sobre avances científicos.
       \item \textbf{Perseverancia: }\\
       Deben ser capaces de superar los obstáculos y seguir adelante hasta completar el proyecto.
       \item[\textit{e. gr.}] Continuación de experimentos a pesar de resultados iniciales no concluyentes.
Rediseño y repetición de experimentos para validar resultados y conclusiones.
      \item \textbf{Flexibilidad: } \\
      Deben ser capaces de adaptarse a los cambios y trabajar de manera flexible.
      \item[\textit{e. gr.}] Adaptación de un proyecto científico debido a cambios inesperados en la disponibilidad de recursos. Flexibilidad en la metodología de investigación ante nuevas evidencias o tecnologías.
      \item \textbf{Confianza en sí mismos:}\\
      Deben creer en sus propias habilidades y capacidades para completar el proyecto.
      \item[\textit{e. gr.}]
      Presentación de descubrimientos a la comunidad científica, defendiendo el trabajo ante críticas constructivas. Asumir desafíos científicos complejos con la confianza de que se pueden alcanzar soluciones efectivas.
    \end{itemize}
\end{itemize}


\newpage
\section{Planificación y Diseño}
\begin{itemize}
  \item \textbf{Semana 5-6: Formulación de Preguntas Guía}
  \begin{itemize}
    \item Sesiones para desarrollar preguntas de investigación que guíen el estudio y la experimentación.
    \item Clarificación de métodos científicos apropiados para abordar estas preguntas.
  \end{itemize}
  \item \textbf{Semana 7-8: Selección de Recursos}
  \begin{itemize}
    \item Compilación de una lista de recursos necesarios, incluyendo equipos de laboratorio, software especializado y bases de datos científicas.
    \item Organización de visitas a laboratorios, centros de investigación o reservas naturales, si aplica.
  \end{itemize}
\end{itemize}

{\large \textbf{Recursos}}\\
\textit{Preguntas:} Lista de preguntas que guían la Planificación y Diseño.

\begin{itemize}
    \item \textbf{Preguntas que profundizan en la investigación y el análisis de datos:}
    \begin{itemize}
        \item ¿Qué información relevante podemos extraer de las imágenes para abordar el problema o reto planteado?
        \item ¿Qué métodos de análisis de datos podemos utilizar para interpretar y extraer patrones de las imágenes?
        \item ¿Cómo podemos combinar diferentes tipos de datos visuales (imágenes, gráficos, mapas) para obtener una comprensión más completa del problema?
        \item ¿Qué herramientas tecnológicas podemos utilizar para procesar, analizar y visualizar datos de imágenes de manera eficiente?
        \item ¿Cómo podemos asegurarnos de que los datos utilizados en nuestro proyecto sean confiables, precisos y relevantes para la investigación?
    \end{itemize}
    \item \textbf{Preguntas que promueven la creatividad y la innovación en la investigación científica:}
    \begin{itemize}
        \item ¿Qué nuevas perspectivas o ideas podemos generar a partir del análisis de imágenes que podrían conducir a soluciones innovadoras?
        \item ¿Cómo podemos utilizar la visualización de datos para comunicar de manera creativa y efectiva nuestros hallazgos a diferentes audiencias?
        \item ¿Cómo podemos integrar el arte, el diseño o la narrativa en nuestro proyecto para presentar los resultados de manera atractiva y memorable?
        \item ¿Qué oportunidades existen para colaborar con expertos en diferentes campos (ciencia de datos, arte, diseño) para enriquecer nuestro proyecto?
        \item ¿Cómo podemos utilizar el proyecto para promover la conciencia pública sobre temas científicos y tecnológicos relevantes para la sociedad?
    \end{itemize}
    \item \textbf{Preguntas que promueven la reflexión y la autoevaluación}
    \begin{itemize}
        \item ¿Cómo ha contribuido nuestro análisis de imágenes a la comprensión del problema o reto científico que investigamos?
        \item ¿Qué habilidades de pensamiento crítico y análisis de datos hemos desarrollado a través de este proyecto?
        \item ¿Cómo ha mejorado nuestra capacidad de trabajar en equipo y colaborar de manera efectiva en un proyecto científico?
        \item ¿Qué desafíos éticos o sociales hemos identificado en el uso de imágenes para la investigación científica?
        \item ¿Cómo podemos aplicar los conocimientos y habilidades adquiridos en este proyecto a futuras investigaciones científicas o proyectos personales?
    \end{itemize}
\end{itemize}

\textit{Wow experience:} Se refiere a una experiencia que es tan impresionante, sorprendente o extraordinaria que provoca una reacción de asombro, admiración o entusiasmo intenso en la persona que la experimenta
\begin{itemize}[label={}]
    \item \textbf{Inmersión Sensorial}
    \begin{itemize}
        \item \textbf{Conceptos Clave:} Estimulación multisensorial
        \item \textbf{Ejemplo Práctico:} Creación de un laboratorio simulado que recrea un ambiente de investigación científica real.
    \end{itemize}
    
    \item \textbf{Desafío Creativo}
    \begin{itemize}
        \item \textbf{Conceptos Clave:} Soluciones innovadoras, pensamiento crítico
        \item \textbf{Ejemplo Práctico:} Diseñar experimentos para testear hipótesis sobre cambios climáticos usando materiales reciclables.
    \end{itemize}
    
    \item \textbf{Interactividad}
    \begin{itemize}
        \item \textbf{Conceptos Clave:} Experimentación activa, manipulación de herramientas científicas
        \item \textbf{Ejemplo Práctico:} Uso de kits de ciencia para realizar experimentos prácticos en pequeños grupos.
    \end{itemize}
    
    \item \textbf{Conexión Emocional}
    \begin{itemize}
        \item \textbf{Conceptos Clave:} Relevancia personal, implicación emocional
        \item \textbf{Ejemplo Práctico:} Proyectos que permitan a los estudiantes explorar temas científicos que afectan directamente a sus comunidades.
    \end{itemize}
    
    \item \textbf{Elemento Sorpresa}
    \begin{itemize}
        \item \textbf{Conceptos Clave:} Descubrimientos inesperados, resultados sorprendentes
        \item \textbf{Ejemplo Práctico:} Revelar resultados de experimentos en una presentación dramática que resalte los descubrimientos más impactantes.
    \end{itemize}
    
    \item \textbf{Reflexión Profunda}
    \begin{itemize}
        \item \textbf{Conceptos Clave:} Análisis crítico, aplicación de conocimientos
        \item \textbf{Ejemplo Práctico:} Sesiones de análisis de datos donde los estudiantes evalúan la validez y las implicaciones de sus experimentos.
    \end{itemize}
\end{itemize}


\newpage
\section{Investigación y Desarollo}
\begin{itemize}
  \item \textbf{Semana 9-16: Investigación Activa}
  \begin{itemize}
    \item Conducción de experimentos, observaciones y recolección de datos según los métodos científicos establecidos.
    \item Análisis de datos e interpretación de resultados, ajustando hipótesis y métodos según sea necesario.
  \end{itemize}
  \item \textbf{Semana 17-24: Desarrollo del Proyecto}
  \begin{itemize}
    \item Aplicación práctica de los hallazgos para construir modelos, prototipos o simulaciones.
    \item Presentaciones periódicas del progreso para discutir con pares y tutores, refinando el enfoque basado en feedback.
  \end{itemize}
\end{itemize}
{\large \textbf{Recursos}}\\
\textit{Visita de Museos:}  Inspira proyectos escolares al fomentar la creatividad, ampliar conocimientos, ofrecer inspiración visual, estimular la investigación, promover la reflexión y propiciar un aprendizaje experiencial.
\begin{itemize}
    \item \textbf{Museo de Historia Natural y Cultura Ambiental}\\
    Av. de los Compositores, Bosque de Chapultepec II Secc, Miguel Hidalgo, 11100 Ciudad de México, CDMX\\
    \url{http://data.sedema.cdmx.gob.mx/museodehistorianatural/?view=featured}
    \item \textbf{Papalote Museo del Niño}\\
    Av Constituyentes 268, Bosque de Chapultepec II Secc, Miguel Hidalgo, 11100 Ciudad de México, CDMX \\
    \url{https://www.papalote.org.mx}

    \item \textbf{Biblioteca Central UNAM} \\
    Escolar S/N, C.U., Coyoacán, 04510 Ciudad de México, CDMX\\
    \url{http://bibliotecacentral.unam.mx/}

    \item \textbf{Biblioteca Vasconcelos} \\ Eje 1 Nte. S/N, Buenavista, Cuauhtémoc, 06350 Ciudad de México, CDMX \\
    \url{http://www.bibliotecavasconcelos.gob.mx/}

    \item \textbf{Biblioteca de México}\\De La Ciudadela 4, Colonia Centro, Centro, Cuauhtémoc, 06040 Ciudad de México, CDMX \\
    \url{http://www.bibliotecademexico.gob.mx/}
    
    \item \textbf{Puertas Abiertas UNAM}\\
    Instituto de Física, UNAM \\
    Noviembre 2024:
    \url{https://www.fisica.unam.mx/puertas_abiertas/pa2024/}
    \item \textbf{Puertas Abiertas Ingenieria UNAM}\\
    Instituto de Ingeniera, UNAM\\
    Noviembre 2024:
    \url{https://www.iingen.unam.mx/es-mx/Difusion/Paginas/puertasAbiertas2023.aspx}
    \item \textbf{Puertas Abiertas Ciencias Nucleares}\\
    Instituto de Ciencias Nucleares, UNAM \\
    \url{https://www.nucleares.unam.mx/dpa_2023/visitas_labs.html}
    \item\textbf{Agencia Espacial Mexicana}\\
    Conferencia con Experto. \\
    \url{https://www.gob.mx/aem}
    \item \textbf{Instituto Nacional de Astrofísica, Óptica y Electrónica (INAOE)}\\
    Luis Enrique Erro No.1, Sta María Tonanzintla, 72840 San Andrés Cholula, Pue.\\
    \url{https://www.inaoep.mx/index.php}

\end{itemize}

\newpage
\section{Creación y Colaboración}
\begin{itemize}
  \item \textbf{Semana 25-26: Trabajo en Equipo}
  \begin{itemize}
    \item Fomento de la colaboración interdisciplinaria, utilizando diversas áreas del conocimiento científico para enriquecer el proyecto.
    \item Evaluación continua del trabajo en equipo y ajuste de roles para optimizar la eficiencia.
  \end{itemize}
  \item \textbf{Semana 27-28: Uso de Tecnología}
  \begin{itemize}
    \item Integración de tecnología avanzada en el proyecto, como software de análisis estadístico, modelado 3D o herramientas de realidad aumentada.
    \item Capacitaciones en el uso eficiente de nuevas tecnologías aplicables al proyecto.
  \end{itemize}
\end{itemize}

{\large \textbf{Recursos}}\\
\textit{Auditorio}: Construcción de un laboratorio (Sesion de carteles). Conferencias dadas por alumnos. Talleres

\begin{itemize}
    \item \textbf{Espacio Físico:}
            \begin{itemize}
                \item \textbf{Aulas Especializadas:} Espacios dedicados separados para experimentos científicos y para robótica, diseñados para facilitar actividades específicas y almacenar equipos relacionados.
                \item \textbf{Superficies de Trabajo Resistentes:} Mesas y bancos resistentes a químicos, cortes y calor, adecuados para experimentos y construcciones de robótica.
            \end{itemize}
    \item \textbf{Equipamiento de Laboratorio:}
            \begin{itemize}
                \item \textbf{Equipos de Medición:} Balanzas, termómetros, manómetros, y otros dispositivos de medición precisos para experimentos científicos.
                \item \textbf{Materiales de Consumo:} Reactivos químicos, soluciones, y kits de experimentación para realizar una variedad de pruebas y experimentos.
            \end{itemize}
    \item \textbf{Recursos de Robótica:}
            \begin{itemize}
                \item \textbf{Kits de Robótica:} Kits modulares de robótica que incluyen motores, sensores, y controladores programables.
                \item \textbf{Herramientas y Equipos de Fabricación:} Herramientas para ensamblaje y modificación de robots, incluyendo soldadores, destornilladores, y cortadores.
            \end{itemize}
    \item \textbf{Tecnología Avanzada:}
            \begin{itemize}
                \item \textbf{Software Especializado:} Programas para diseño y simulación en 3D, software de programación para robots, y aplicaciones para análisis de datos experimentales.
                \item \textbf{Impresoras 3D:} Para prototipado rápido de piezas y componentes utilizados en robótica y experimentos científicos.
            \end{itemize}
    \item \textbf{Seguridad:}
            \begin{itemize}
                \item \textbf{Equipos de Seguridad:} Gafas de seguridad, guantes, batas de laboratorio, y otros equipos de protección personal.
                \item \textbf{Ventilación Adecuada:} Sistemas de extracción y ventilación para manejar vapores y gases peligrosos.
            \end{itemize}
    \item \textbf{Almacenamiento Seguro:}
            \begin{itemize}
                \item \textbf{Armarios para Químicos:} Armarios ventilados y a prueba de fuego para almacenar sustancias químicas peligrosas.
                \item \textbf{Almacenamiento de Herramientas y Robots:} Espacios designados y seguros para guardar herramientas y robots cuando no estén en uso.
            \end{itemize}
    \item \textbf{Espacio de Colaboración:}
            \begin{itemize}
                \item \textbf{Áreas de Trabajo en Equipo:} Espacios diseñados para facilitar la colaboración entre los alumnos en proyectos de grupo.
                \item \textbf{Pizarras Interactivas:} Para facilitar la discusión grupal y la planificación de proyectos.
            \end{itemize}
    \item \textbf{Conectividad:}
            \begin{itemize}
                \item \textbf{Acceso a Internet de Alta Velocidad:} Para investigación, descarga de programas y acceso a recursos en línea.
                \item \textbf{Red Local Segura:} Para conectar todos los dispositivos y equipos de forma segura dentro de la sala.
            \end{itemize}
\end{itemize}




\newpage
\section{Prentación y Evaluación}
\begin{itemize}
  \item \textbf{Semana 29-30: Presentación del Proyecto}
  \begin{itemize}
    \item Preparación de informes científicos y presentaciones para comunicar los resultados del proyecto.
    \item Simulación de conferencias científicas para presentar los hallazgos a una audiencia crítica.
  \end{itemize}
  \item \textbf{Semana 31-32: Reflexión y Evaluación}
  \begin{itemize}
    \item Revisión crítica de todo el proceso de investigación y los resultados obtenidos.
    \item Evaluaciones formativas y sumativas, incluyendo autoevaluaciones y evaluaciones por pares.
  \end{itemize}
\end{itemize}
\begin{itemize}
    \item \textbf{Tipos de Presentación (Performance)}
    \begin{itemize}
        \item \textbf{Demostración Experimental:}
        Los estudiantes realizan experimentos en vivo frente a una audiencia, demostrando técnicas científicas y los resultados de sus investigaciones.
        \item \textbf{Presentación de Póster:}
        Creación de pósteres científicos que resumen el proyecto de investigación, incluyendo hipótesis, metodología, resultados y conclusiones.
        \item \textbf{Conferencia Estudiantil:}
        Presentaciones formales en las que los estudiantes exponen sus hallazgos en un formato similar al de conferencias profesionales científicas.
        \item \textbf{Video Documental:}
        Producción de un video que documenta todo el proceso del proyecto, desde la formulación de la pregunta de investigación hasta las conclusiones finales.
        \item \textbf{Simulaciones Computarizadas:}
        Uso de software para simular experimentos o fenómenos científicos que son difíciles de reproducir en un laboratorio escolar.
    \end{itemize}
    \item \textbf{Tipos de Evaluación (Rúbricas)}
    \begin{itemize}
        \item \textbf{Rúbrica de Metodología Científica:}
        Evalúa la precisión y adecuación de las metodologías empleadas en la investigación, incluyendo el diseño experimental y el uso de controles.
        \item \textbf{Rúbrica de Análisis de Datos:}
        Centrada en la habilidad de analizar y interpretar datos científicos, incluyendo el uso de estadísticas y la representación gráfica de resultados.
        \item \textbf{Rúbrica de Innovación y Originalidad:}
        Evalúa la capacidad del estudiante para aportar ideas originales y creativas al campo de estudio, así como la aplicación de soluciones innovadoras a problemas científicos.
        \item \textbf{Rúbrica de Comunicación Científica:}
        Mide la claridad y efectividad con la que los estudiantes comunican sus hallazgos, tanto oralmente como por escrito, utilizando el lenguaje científico apropiado.
        \item \textbf{Rúbrica de Impacto y Aplicabilidad:}
        Evalúa la relevancia y el potencial impacto de los resultados del proyecto en el campo científico o en la solución de problemas prácticos.
    \end{itemize}
\end{itemize}




\newpage
\section{Reflexión y Retroalimentación}
\begin{itemize}
  \item \textbf{Semana 33-34: Retroalimentación Continua}
  \begin{itemize}
    \item Análisis detallado del feedback recibido durante las presentaciones.
    \item Sesiones de mejora basadas en las críticas y recomendaciones de expertos.
  \end{itemize}
  \item \textbf{Semana 35-36: Reflexión Final}
  \begin{itemize}
    \item Reflexiones sobre el aprendizaje científico adquirido y su aplicación futura.
    \item Planificación de publicaciones o participaciones en ferias de ciencias para compartir los resultados del proyecto con una audiencia más amplia.
  \end{itemize}
\end{itemize}
{\large \textbf{Recursos}}\\
\textit{Actividades: } Performativos para la reflexión y retroalimentación activa para el estudiante.  
\begin{itemize}
    \item \textbf{Diarios de Laboratorio:} Los estudiantes mantienen un diario de laboratorio detallado donde registran cada paso de sus experimentos, observaciones y reflexiones. Este diario se comparte y discute periódicamente con compañeros y profesores para obtener retroalimentación.
    \item \textbf{Presentaciones de Investigación:} Los estudiantes crean presentaciones detalladas de sus proyectos de investigación, que incluyen metodología, datos, análisis y conclusiones. Estas presentaciones se hacen ante la clase y un panel de profesores que proporcionan críticas constructivas.
    \item \textbf{Sesiones de Póster:} Organización de sesiones de pósteres donde los estudiantes exhiben sus hallazgos en formato de póster científico, facilitando un diálogo interactivo con observadores y otros estudiantes.
    \item \textbf{Mesas Redondas de Discusión:} Formación de grupos de discusión para hablar sobre temas específicos relacionados con los proyectos. Estas sesiones fomentan el intercambio de ideas y permiten a los estudiantes defender sus métodos y resultados.
    \item \textbf{Revisiones por Pares:} Implementación de un proceso de revisión por pares, donde los estudiantes evalúan el trabajo de otros basándose en criterios establecidos, proporcionando y recibiendo retroalimentación estructurada.
    \item \textbf{Simulaciones de Conferencias:} Simulación de conferencias científicas donde los estudiantes presentan sus investigaciones y reciben preguntas y comentarios de un público informado, incluyendo profesores y estudiantes avanzados.
    \item \textbf{Blogs de Proyectos:} Creación de blogs o vlogs donde los estudiantes publican regularmente sobre el progreso de sus proyectos, recibiendo comentarios del público y de expertos en el área.
    \item \textbf{Evaluaciones Interactivas:} Uso de tecnologías interactivas para que los estudiantes respondan a cuestionarios o realicen autoevaluaciones sobre su aprendizaje y el de sus compañeros.
    \item \textbf{Foros en Línea:} Participación en foros de discusión en línea donde los estudiantes pueden publicar sobre sus proyectos, obtener retroalimentación de una comunidad más amplia y discutir problemas científicos relevantes.
    \item \textbf{Talleres de Crítica Constructiva:} Talleres diseñados para enseñar y practicar la crítica constructiva, donde los estudiantes aprenden a dar y recibir retroalimentación efectiva que fomente el crecimiento académico y personal.
\end{itemize}

\newpage
\section{Materiales}
\textbf{Materiales de papelería}

\begin{itemize}
  \item \textbf{Papeles Especiales}
  \begin{itemize}
      \item Papel milimetrado para gráficos y datos.
      \item Papel para diagramas y esquemas.
      \item Papel para impresiones de alta calidad para reportes y presentaciones.
  \end{itemize}
  
  \item \textbf{Herramientas de Dibujo y Medición}
  \begin{itemize}
      \item Lápices, bolígrafos y rotuladores para etiquetar y dibujar.
      \item Reglas, compases y transportadores para mediciones precisas.
      \item Calculadoras cinetíficas.
  \end{itemize}
  
  \item \textbf{Modelado y Prototipado}
  \begin{itemize}
      \item Kits de construcción para modelos físicos de estructuras o máquinas.
      \item Materiales como plastilina o arcilla para modelar estructuras.
  \end{itemize}
  
  \item \textbf{Instrumentos de Corte}
  \begin{itemize}
      \item Tijeras de precisión.
      \item Cuchillas y cortadores para modelos de papel y cartón.
      \item Guillotinas para cortar múltiples hojas de papel.
  \end{itemize}
  
  \item \textbf{Adhesivos y Sujeciones}
  \begin{itemize}
      \item Pegamentos líquidos para ensamblajes permanentes.
      \item Cintas adhesivas variadas para fijar temporalmente o permanentemente.
      \item Velcro y tiras magnéticas para montajes y demostraciones.
  \end{itemize}
  
  \item \textbf{Equipos de Documentación}
  \begin{itemize}
      \item Cámaras de documentos para capturar experimentos en tiempo real.
      \item Grabadoras de voz para tomar notas o registrar observaciones.
      \item Tablets o dispositivos digitales para recopilar y analizar datos.
  \end{itemize}
\end{itemize}