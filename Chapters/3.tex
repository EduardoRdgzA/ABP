\section{Justificación}
En el contexto del \textbf{Aprendizaje Personalizado}, los proyectos orientados a las artes en primer grado de secundaria representan una herramienta fundamental para el desarrollo integral del estudiante. Estos proyectos no solo facilitan el \textbf{Aprendizaje Activo y para la Vida} mediante la aplicación directa de conocimientos artísticos en contextos reales, sino que también promueven la inclusión y la accesibilidad, pilares educativos clave en un entorno de \textbf{Neuroeducación}. Al integrar la \textbf{Tecnología Educativa} en las artes, los estudiantes exploran nuevas formas de expresión y creación, lo que fomenta un aprendizaje más profundo y significativo.
\\ \\ 
El \textbf{Perfil de Egreso} de estudiantes que participan en proyectos basados en las artes incluye competencias en \textbf{Artes} y la \textbf{Autogestión del Aprendizaje para la Vida}, elementos esenciales del \textbf{Humanismo, Acción y Servicio}. A través de estos proyectos, los alumnos desarrollan habilidades en \textbf{Trabajo en Equipo y Comunicación}, fundamentales para su desempeño futuro tanto en ámbitos personales como profesionales. Estos proyectos refuerzan los \textbf{Valores y Self-System} del estudiante, proporcionando un marco sólido que les permite enfrentar desafíos futuros con creatividad, responsabilidad y eficacia.
\\ \\ 
\textbf{Motivos Suficientes:} \\
\begin{itemize}
    \item \textit{Fomento de la creatividad y expresión personal:}
    A través del arte, los estudiantes pueden explorar y expresar sus emociones, ideas y perspectivas, desarrollando a la vez su creatividad y pensamiento crítico.
    \item \textit{Desarrollo de habilidades transversales:}
    Las artes integran habilidades como la colaboración, la comunicación efectiva y la resolución de problemas, esenciales en cualquier ámbito profesional.
\end{itemize}
\textbf{Motivos Necesarios:}
\begin{itemize}
    \item \textit{Inclusión y accesibilidad:}
     Las artes permiten que estudiantes de diversos intereses y capacidades se involucren activamente, promoviendo un ambiente inclusivo.
     \item \textit{Aplicación de conocimientos en contextos reales:}
    El arte como proyecto permite aplicar conocimientos teóricos en creaciones prácticas, consolidando el aprendizaje mediante la experimentación directa.
\end{itemize}

\newpage
\section{Inicio del Proyecto}
\begin{itemize}
  \item \textbf{Semana 1-2: Presentación del Desafío o Problema Central}
  \begin{itemize}
    \item Introducción a un desafío de negocio real, como el lanzamiento de un nuevo producto, la mejora de un servicio existente, o la solución a un problema de mercado.
    \item Sesiones de brainstorming para explorar necesidades del mercado y oportunidades de innovación.
  \end{itemize}
  \item \textbf{Semana 3-4: Definición de Objetivos de Aprendizaje}
  \begin{itemize}
    \item Talleres para identificar habilidades clave en emprendimiento, como planificación estratégica, análisis financiero y marketing.
    \item Elaboración de un marco de competencias empresariales que los estudiantes deberán desarrollar.
  \end{itemize}
\end{itemize}
{\large \textbf{Recursos}}\\
\textit{Condiciones Iniciales:} Requirimientos para los alumnos puedan llevar a cabo el Aprendizaje Basado en Proyectos (ABP):
\begin{itemize}
    \item \textbf{Habilidades cognitivas:}
    \begin{itemize}
        \item \textbf{Pensamiento crítico:}\\
        Los alumnos deben ser capaces de analizar información, identificar problemas, formular hipótesis y evaluar soluciones.
        \item[\textit{e. gr.}]Análisis de estudios de mercado para identificar nichos rentables y necesidades no satisfechas.Evaluación crítica de modelos de negocio existentes para mejorarlos o adaptarlos a nuevos mercados.
        \item \textbf{Creatividad: }\\
        Deben ser capaces de generar ideas originales y pensar de manera innovadora para encontrar soluciones a los problemas.
        \item[\textit{e. gr.}]Desarrollo de un producto innovador que soluciona un problema común de manera no tradicional. Creación de una campaña de marketing original que capte la atención de un público objetivo específico.
        \item \textbf{Resolución de problemas:}\\
        Deben ser capaces de aplicar sus conocimientos y habilidades para resolver problemas de manera efectiva.
        \item[\textit{e. gr.}]Diseño de estrategias para superar barreras de entrada en mercados altamente competitivos. Implementación de soluciones tecnológicas para mejorar la eficiencia operativa de un nuevo negocio.
        \item \textbf{Toma de decisiones:} \\
        Deben ser capaces de evaluar diferentes opciones y tomar decisiones informadas.
        \item[\textit{e. gr.}]Selección de proveedores y socios estratégicos que alineen con la visión y valores del emprendimiento. Decisión sobre la estructura de financiamiento más adecuada para el crecimiento sostenible del negocio.
        \item \textbf{Metacognición:}\\
        Deben ser capaces de reflexionar sobre su propio proceso de aprendizaje y identificar sus fortalezas y debilidades.
         \item[\textit{e. gr.}]
        Reflexión sobre el propio estilo de liderazgo y su efectividad en diferentes etapas del emprendimiento. Autoevaluación de habilidades empresariales y determinación de áreas para desarrollo profesional continuo.
    \end{itemize}
    \item \textbf{Habilidades sociales:}
    \begin{itemize}
        \item \textbf{Trabajo en equipo: }\\
        Los alumnos deben ser capaces de trabajar de manera colaborativa con otros para alcanzar un objetivo común.
        \item[\textit{e. gr.}] Formación de un equipo fundador diverso que aporte habilidades complementarias al emprendimiento. Organización de sesiones de brainstorming con el equipo para generar ideas de productos o servicios.
        \item \textbf{Comunicación: }\\
        Deben ser capaces de comunicar sus ideas de manera efectiva, tanto oralmente como por escrito.
        \item[\textit{e. gr.}]Elaboración y presentación de pitches a inversores potenciales. Comunicación efectiva con clientes y stakeholders para construir relaciones duraderas.
        \item \textbf{Liderazgo:}\\
        Deben ser capaces de asumir roles de liderazgo y motivar a otros a trabajar en equipo.
        \item[\textit{e. gr.}]Dirección de un equipo durante el lanzamiento de un startup, manteniendo la motivación alta bajo presión. Mentoría a miembros más jóvenes del equipo, proporcionando guía y fomentando su crecimiento profesional.
        \item \textbf{Empatía:}\\
         Deben ser capaces de comprender y respetar los puntos de vista de los demás.
         \item[\textit{e. gr.}] Desarrollo de productos o servicios que atiendan las necesidades específicas de comunidades desatendidas. Implementación de prácticas de negocio éticas que consideren el bienestar de empleados y la comunidad.
         \item \textbf{Responsabilidad: }\\
         Deben ser responsables de su propio aprendizaje y del trabajo que realizan en equipo.
         \item[\textit{e. gr.}]
         Cumplimiento riguroso con regulaciones y leyes aplicables al lanzar y operar el negocio. Gestión responsable de los recursos financieros del emprendimiento para asegurar su viabilidad a largo plazo.
    \end{itemize}
    \item \textbf{Habilidades técnicas:}
    \begin{itemize}
        \item \textbf{Investigación:}\\
        Los alumnos deben ser capaces de investigar información de manera efectiva utilizando diferentes fuentes.
        \item[\textit{e. gr.}]Realización de encuestas y focus groups para validar la idea del negocio y ajustar el enfoque del producto. Investigación continua sobre tendencias del mercado y tecnologías emergentes que puedan impactar el negocio.
        \item \textbf{Manejo de información:}\\
         Deben ser capaces de organizar y analizar información de manera eficiente.
         \item[\textit{e. gr.}]Uso de herramientas de análisis de datos para interpretar información del mercado y guiar decisiones estratégicas. Organización y protección de la propiedad intelectual generada por el emprendimiento.
         \item \textbf{Uso de tecnología:}\\
         Deben ser capaces de utilizar las herramientas tecnológicas de manera adecuada para apoyar su aprendizaje.
         \item[\textit{e. gr.}] Implementación de plataformas de comercio electrónico para la venta de productos o servicios. Uso de software de gestión de proyectos para coordinar y monitorear el progreso del emprendimiento.
         \item \textbf{Gestión del tiempo:}\\
         Deben ser capaces de planificar y gestionar su tiempo de manera efectiva para completar las tareas del proyecto.
         \item[\textit{e. gr.}]Planificación detallada de las etapas de desarrollo y lanzamiento del producto para cumplir con los plazos del mercado. Priorización eficiente de tareas críticas para el éxito del negocio.
    \end{itemize}
    \item \textbf{Actitudes:}
    \begin{itemize}
        \item \textbf{Motivación:}\\
        Los alumnos deben estar motivados para aprender y participar activamente en el proyecto.
        \item[\textit{e. gr.}]Participación en competiciones de startups para obtener financiación y visibilidad. Búsqueda activa de retroalimentación y mentoría para mejorar continuamente el modelo de negocio.
        \item \textbf{Interés:}\\
        Deben estar interesados en el tema del proyecto.
        \item[\textit{e. gr.}] Asistencia regular a eventos y talleres de emprendimiento para ampliar la red de contactos y aprender de otros emprendedores. Investigación y desarrollo de un profundo conocimiento del sector en el que opera el negocio.
       \item \textbf{Perseverancia: }\\
       Deben ser capaces de superar los obstáculos y seguir adelante hasta completar el proyecto.
       \item[\textit{e. gr.}] Superación de rechazos iniciales de inversores y continuación en la búsqueda de financiamiento. Resolución de problemas inesperados durante el desarrollo del producto sin perder el enfoque en los objetivos a largo plazo.
      \item \textbf{Flexibilidad: } \\
      Deben ser capaces de adaptarse a los cambios y trabajar de manera flexible.
      \item[\textit{e. gr.}] Adaptación del modelo de negocio ante cambios en el entorno económico o tecnológico. Flexibilidad para pivotar el enfoque del producto basado en la retroalimentación del cliente y el desempeño del mercado.
      \item \textbf{Confianza en sí mismos:}\\
      Deben creer en sus propias habilidades y capacidades para completar el proyecto.
      \item[\textit{e. gr.}]
      Defensa apasionada de la visión del negocio frente a escepticismo o críticas. Toma de decisiones audaces que definan la dirección futura del emprendimiento.
    \end{itemize}
\end{itemize}



\newpage
\section{Planificación y Diseño}
\begin{itemize}
  \item \textbf{Semana 5-6: Formulación de Preguntas Guía}
  \begin{itemize}
    \item Desarrollo de preguntas clave que guiarán la creación y validación de modelos de negocio.
    \item Enfoque en metodologías ágiles y diseño de experimentos para testear ideas rápidamente.
  \end{itemize}
  \item \textbf{Semana 7-8: Selección de Recursos}
  \begin{itemize}
    \item Identificación de herramientas y recursos esenciales para emprendedores, como software de gestión de proyectos, plataformas de análisis de mercado y recursos de financiación.
    \item Conexión con mentores y redes de emprendedores para facilitar el acceso a asesoramiento y apoyo.
  \end{itemize}
\end{itemize}
{\large \textbf{Recursos}}\\
\textit{Preguntas:} Lista de preguntas que guían la Planificación y Diseño.
\begin{itemize}
    \item \textbf{Preguntas que profundizan en la comprensión del mercado y la viabilidad de la idea:}
    \begin{itemize}
        \item ¿Qué necesidades o problemas específicos del mercado podemos identificar y abordar a través de nuestro emprendimiento basado en imágenes?
        \item ¿A qué segmento de mercado nos dirigiremos y cómo podemos llegar a nuestro público objetivo de manera efectiva?
        \item ¿Qué análisis de la competencia podemos realizar para comprender el panorama del mercado y diferenciar nuestro producto o servicio?
        \item ¿Qué análisis de la competencia podemos realizar para comprender el panorama del mercado y diferenciar nuestro producto o servicio?
        \item ¿Cómo podemos utilizar la información de la imagen para crear una propuesta de valor única y atractiva para nuestros clientes?
        \item ¿Qué estrategias de marketing y comunicación podemos implementar para dar a conocer nuestro emprendimiento y generar interés en los consumidores?
    \end{itemize}
    \item \textbf{Preguntas que fomentan la creatividad y la innovación en el desarrollo del producto o servicio:}
    \begin{itemize}
        \item ¿Cómo podemos utilizar la información de la imagen de manera creativa para diseñar un producto o servicio innovador y atractivo?
       \item ¿Qué tecnologías emergentes o herramientas digitales podemos incorporar para mejorar la experiencia del cliente y agregar valor a nuestro emprendimiento?
       \item ¿Cómo podemos aprovechar el poder de las redes sociales y el marketing visual para promocionar nuestro emprendimiento y llegar a una audiencia global?
       \item ¿Qué oportunidades existen para colaborar con artistas, diseñadores o fotógrafos para enriquecer la propuesta visual de nuestro emprendimiento?
       \item ¿Cómo podemos asegurarnos de que nuestro emprendimiento basado en imágenes sea sostenible, responsable con el medio ambiente y alineado con nuestros valores éticos?
    \end{itemize}
    \item \textbf{Preguntas que promueven la reflexión y la autoevaluación}
    \begin{itemize}
        \item ¿Cómo ha contribuido nuestro análisis de imágenes a la definición de la propuesta de valor de nuestro emprendimiento?
        \item ¿Qué habilidades de pensamiento creativo, resolución de problemas y toma de decisiones hemos desarrollado a través de este proyecto?
        \item ¿Cómo ha mejorado nuestra capacidad de trabajar en equipo, colaborar de manera efectiva y gestionar proyectos de manera eficiente?
        \item ¿Qué desafíos éticos o sociales hemos identificado en el desarrollo de nuestro emprendimiento basado en imágenes?
        \item ¿Cómo podemos aplicar los conocimientos y habilidades adquiridos en este proyecto a futuras iniciativas emprendedoras o proyectos personales?
    \end{itemize}
\end{itemize}
\textit{Wow experience:} Se refiere a una experiencia que es tan impresionante, sorprendente o extraordinaria que provoca una reacción de asombro, admiración o entusiasmo intenso en la persona que la experimenta
\begin{itemize}[label={}]
    \item \textbf{Inmersión Sensorial}
    \begin{itemize}
        \item \textbf{Conceptos Clave:} Estimulación ambiental realista
        \item \textbf{Ejemplo Práctico:} Simulación de un entorno de mercado o feria de emprendedores donde los estudiantes presentan sus proyectos.
    \end{itemize}
    
    \item \textbf{Desafío Creativo}
    \begin{itemize}
        \item \textbf{Conceptos Clave:} Innovación, solución de problemas
        \item \textbf{Ejemplo Práctico:} Crear un producto o servicio que solucione un problema específico de su comunidad.
    \end{itemize}
    
    \item \textbf{Interactividad}
    \begin{itemize}
        \item \textbf{Conceptos Clave:} Participación activa, feedback del usuario
        \item \textbf{Ejemplo Práctico:} Interacciones con clientes potenciales para obtener retroalimentación y mejorar el producto.
    \end{itemize}
    
    \item \textbf{Conexión Emocional}
    \begin{itemize}
        \item \textbf{Conceptos Clave:} Pasión por el proyecto, compromiso personal
        \item \textbf{Ejemplo Práctico:} Proyectos que se alineen con las pasiones personales de los estudiantes, fomentando un mayor compromiso.
    \end{itemize}
    
    \item \textbf{Elemento Sorpresa}
    \begin{itemize}
        \item \textbf{Conceptos Clave:} Resultados inesperados, oportunidades emergentes
        \item \textbf{Ejemplo Práctico:} Invitar a un emprendedor exitoso como mentor sorpresa para dar consejos y guiar el desarrollo del proyecto.
    \end{itemize}
    
    \item \textbf{Reflexión Profunda}
    \begin{itemize}
        \item \textbf{Conceptos Clave:} Evaluación crítica, escalabilidad del negocio
        \item \textbf{Ejemplo Práctico:} Sesiones de revisión donde los estudiantes analizan la viabilidad y escalabilidad de su emprendimiento.
    \end{itemize}
\end{itemize}




\newpage
\section{Investigación y Desarollo}
\begin{itemize}
  \item \textbf{Semana 9-16: Investigación Activa}
  \begin{itemize}
    \item Investigación de mercado para validar la demanda del producto o servicio propuesto.
    \item Desarrollo iterativo del modelo de negocio basado en feedback continuo de potenciales clientes y mentores.
  \end{itemize}
  \item \textbf{Semana 17-24: Desarrollo del Proyecto}
  \begin{itemize}
    \item Implementación y refinamiento de prototipos o versiones beta del producto o servicio.
    \item Pruebas de mercado y ajustes basados en análisis de datos y feedback de usuarios.
  \end{itemize}
\end{itemize}

{\large \textbf{Recursos}}\\
\textit{Visita de Museos:}  Inspira proyectos escolares al fomentar la creatividad, ampliar conocimientos, ofrecer inspiración visual, estimular la investigación, promover la reflexión y propiciar un aprendizaje experiencial.
\begin{itemize}
    \item \textbf{MIDE, Museo Interactivo de Economía} \\ C. de Tacuba 17, Centro Histórico de la Cdad. de México, Centro, Cuauhtémoc, 06000 Ciudad de México, CDMX \\
    \url{https://www.mide.org.mx/}
    \item \textbf{MUBO, Museo de la Bolsa Mexicana de Valores}\\
    Av. P.º de la Reforma 255-mezzanine, Cuauhtémoc, 06500 Ciudad de México, CDMX \\
    \url{https://www.mubo.com.mx/pages/default}
    \item \textbf{Planta Bimbo Azcapotzalco} \\ 
    Benito Juárez \# 111, Col. Reynosa Tamaulipas C.p. 02200 , Alcaldia Azcapotzalco, Azcapotzalco, CDMX \\
    Agendar cita:  
    \url{https://visitas.bimboconnect.com/#formulario}
\end{itemize}


\newpage
\section{Creación y Colaboración}
\begin{itemize}
  \item \textbf{Semana 25-26: Trabajo en Equipo}
  \begin{itemize}
    \item Desarrollo de habilidades de liderazgo y gestión de equipos, crucial para el éxito empresarial.
    \item Asignación de roles dentro del equipo que reflejen las fortalezas y habilidades de cada miembro.
  \end{itemize}
  \item \textbf{Semana 27-28: Uso de Tecnología}
  \begin{itemize}
    \item Integración de herramientas tecnológicas avanzadas para la gestión del negocio, marketing digital y operaciones.
    \item Talleres sobre herramientas específicas de e-commerce, marketing digital o análisis financiero.
  \end{itemize}
\end{itemize}
{\large \textbf{Recursos}}\\
\textit{Auditorio}: Simulación de Mercado

\begin{itemize}
    \item \textbf{Espacio Físico:}
            \begin{itemize}
                \item \textbf{Área de Mercado:} Un espacio amplio como un gimnasio, patio escolar o aula grande que se pueda adaptar para simular un ambiente de mercado.
                \item \textbf{Stands y Puestos:} Estructuras temporales o móviles donde los estudiantes puedan exhibir y vender sus productos o servicios.
            \end{itemize}
    \item \textbf{Mobiliario:}
            \begin{itemize}
                \item \textbf{Mesas y Sillas:} Para la exhibición de productos y para que los estudiantes manejen sus transacciones.
                \item \textbf{Tableros de Anuncios:} Para precios, ofertas y publicidad de los productos disponibles.
            \end{itemize}
    \item \textbf{Materiales de Marketing:}
            \begin{itemize}
                \item \textbf{Material Impreso:} Folletos, tarjetas de visita y carteles para promocionar los productos o servicios.
                \item \textbf{Etiquetas y Precios:} Para marcar claramente los productos con su precio correspondiente.
            \end{itemize}
    \item \textbf{Equipamiento Tecnológico:}
            \begin{itemize}
                \item \textbf{Sistemas de Pago:} Terminales para pagos con tarjeta o aplicaciones móviles para facilitar transacciones sin efectivo.
                \item \textbf{Software de Simulación:} Programas que puedan simular variaciones de mercado, demanda y logística de suministro.
            \end{itemize}
    \item \textbf{Recursos Didácticos:}
            \begin{itemize}
                \item \textbf{Juegos de Rol:} Actividades donde los estudiantes asumen roles de compradores, vendedores, reguladores del mercado, etc.
                \item \textbf{Guías de Aprendizaje:} Documentos o manuales que proporcionan instrucciones sobre economía básica y habilidades de venta.
            \end{itemize}
    \item \textbf{Señalización:}
            \begin{itemize}
                \item \textbf{Señales Informativas:} Carteles y señales que ayudan a dirigir el tráfico dentro del mercado y proporcionan información sobre los productos.
                \item \textbf{Áreas Designadas:} Señalización para diferentes secciones del mercado, como alimentación, tecnología, ropa, etc.
            \end{itemize}
    \item \textbf{Medidas de Seguridad:}
            \begin{itemize}
                \item \textbf{Plan de Evacuación:} Protocolos de seguridad en caso de emergencia, claramente comunicados y señalizados.
                \item \textbf{Primeros Auxilios:} Un kit de primeros auxilios accesible para manejar cualquier incidente menor.
            \end{itemize}
\end{itemize}





\newpage
\section{Prentación y Evaluación}
\begin{itemize}
  \item \textbf{Semana 29-30: Presentación del Proyecto}
  \begin{itemize}
    \item Preparación de un pitch de negocios para presentar ante inversores, expertos en la industria o potenciales clientes.
    \item Simulaciones de negociaciones y reuniones con stakeholders para afinar la estrategia de comunicación.
  \end{itemize}
  \item \textbf{Semana 31-32: Reflexión y Evaluación}
  \begin{itemize}
    \item Análisis del proceso empresarial y evaluación del potencial comercial del proyecto.
    \item Evaluaciones tanto formativas como sumativas para medir el impacto y la viabilidad del negocio propuesto.
  \end{itemize}
\end{itemize}

{\large \textbf{Recursos}}
\begin{itemize}
    \item \textbf{Tipos de Presentación (Performance)}
    \begin{itemize}
        \item \textbf{Pitch de Negocios:}
        Los estudiantes presentan su idea de negocio en un formato de pitch breve, destacando el problema que resuelven, la solución propuesta, el mercado objetivo y el modelo de negocio.
        \item \textbf{Feria de Emprendimiento:}
        Creación de stands donde los estudiantes pueden demostrar sus productos o servicios y interactuar directamente con potenciales clientes o inversores.
        \item \textbf{Demostración de Producto:}
        Presentación en vivo del prototipo o producto final, mostrando su funcionalidad y beneficios para el usuario final.
        \item \textbf{Simulación de Mercado:}
        Simulación de un entorno de mercado donde los estudiantes pueden vender sus productos o servicios y analizar la respuesta del mercado.
        \item \textbf{Informe de Viabilidad:}
        Presentación de un informe detallado que analiza la viabilidad técnica, comercial y financiera del proyecto emprendedor.
    \end{itemize}
    \item \textbf{Tipos de Evaluación (Rúbricas)}
    \begin{itemize}
        \item \textbf{Rúbrica de Viabilidad Comercial:}
        Evalúa la claridad y la viabilidad del modelo de negocio propuesto, incluyendo análisis de mercado, estrategia de precios y proyecciones financieras.
        \item \textbf{Rúbrica de Innovación y Creatividad:}
        Centrada en la originalidad de la idea de negocio y la creatividad en la solución de problemas y en el diseño del producto o servicio.
        \item \textbf{Rúbrica de Planificación y Ejecución:}
        Evalúa la capacidad de planificar y gestionar recursos para llevar a cabo el proyecto, incluyendo la gestión del tiempo y del equipo.
        \item \textbf{Rúbrica de Comunicación y Marketing:}
        Mide la efectividad de las estrategias de comunicación y marketing utilizadas para promover el producto o servicio entre el público objetivo.
        \item \textbf{Rúbrica de Impacto Social o Ambiental:}
        Evalúa la contribución del proyecto al bienestar social o la mejora del medio ambiente, considerando aspectos como sostenibilidad y responsabilidad social.
    \end{itemize}
\end{itemize}




\newpage
\section{Reflexión y Retroalimentación}
\begin{itemize}
  \item \textbf{Semana 33-34: Retroalimentación Continua}
  \begin{itemize}
    \item Recepción y análisis de comentarios de expertos y asesores post-presentación.
    \item Ajustes en el plan de negocio y estrategias de salida o expansión basados en el feedback recibido.
  \end{itemize}
  \item \textbf{Semana 35-36: Reflexión Final}
  \begin{itemize}
    \item Sesiones de cierre para reflexionar sobre el aprendizaje y las habilidades empresariales adquiridas.
    \item Discusión sobre futuras oportunidades de emprendimiento y la transferencia de habilidades a nuevos proyectos.
  \end{itemize}
\end{itemize}

{\large \textbf{Recursos}}\\
\textit{Actividades: } Performativos para la reflexión y retroalimentación activa para el estudiante.  


\begin{itemize}
    \item \textbf{Sesiones de Pitch:} Los estudiantes presentan su idea de negocio a un panel de expertos simulando un escenario de inversión. Reciben comentarios directos sobre la claridad, viabilidad y atractivo de su propuesta.
    \item \textbf{Talleres de Mejora Continua:} Talleres donde los estudiantes revisan y discuten sus modelos de negocio basándose en retroalimentación previa, aprendiendo a iterar y mejorar sus propuestas.
    \item \textbf{Análisis de Casos de Estudio:} Discusión y análisis en grupo de estudios de caso relevantes al proyecto, permitiendo a los estudiantes comparar y contrastar sus propios procesos y resultados con ejemplos reales de éxito y fracaso.
    \item \textbf{Mesas Redondas con Emprendedores:} Encuentros con emprendedores reales donde los estudiantes presentan sus proyectos y discuten desafíos y estrategias, ganando insights prácticos y retroalimentación crítica.
    \item \textbf{Diarios de Emprendimiento:} Mantenimiento de un diario o blog por cada estudiante donde se documentan los avances, los obstáculos encontrados y las lecciones aprendidas, facilitando la reflexión personal y grupal.
    \item \textbf{Presentaciones de Avance:} Presentaciones periódicas del progreso del proyecto ante la clase, permitiendo a los estudiantes articular su aprendizaje y ajustar sus planes basándose en preguntas y comentarios de compañeros y profesores.
    \item \textbf{Simulaciones de Negociación:} Actividades donde los estudiantes practican negociaciones con "clientes" o "inversores" ficticios, recibiendo retroalimentación sobre sus habilidades de comunicación y persuasión.
    \item \textbf{Evaluaciones de Impacto:} Sesiones donde los estudiantes evalúan el impacto potencial de su negocio en la sociedad o el medio ambiente, recibiendo retroalimentación sobre la responsabilidad social de su emprendimiento.
    \item \textbf{Revisiones de Planes de Negocio:} Revisiones detalladas de los planes de negocio escritos, donde los estudiantes reciben comentarios específicos sobre cada sección del plan, ayudando a identificar fortalezas y áreas de mejora.
    \item \textbf{Ferias de Emprendimiento Interno:} Eventos donde los estudiantes exponen y venden sus productos o servicios dentro de la escuela, obteniendo retroalimentación directa del mercado sobre la aceptación y el interés en sus ofertas.
\end{itemize}

\newpage
\section{Materiales}
\textbf{Materiales de papelería}
\begin{itemize}
  \item \textbf{Materiales de Oficina Básicos}
  \begin{itemize}
      \item Papel de varios tipos (blanco, reciclado, de colores).
      \item Carpetas y organizadores para documentos.
      \item Clips, grapas, y otros sujeta-papeles.
  \end{itemize}
  
  \item \textbf{Herramientas de Planificación y Organización}
  \begin{itemize}
      \item Agendas y planificadores para gestión de proyectos.
      \item Tableros blancos y de corcho para brainstorming y seguimiento de tareas.
      \item Post-its y notas adhesivas para anotaciones rápidas y organización visual.
  \end{itemize}
  
  \item \textbf{Herramientas de Presentación}
  \begin{itemize}
      \item Cartulinas y papelógrafos para presentaciones manuales.
      \item Marcadores de varios colores y rotuladores para pizarras.
      \item Proyectores y pantallas para presentaciones digitales.
  \end{itemize}
  
  \item \textbf{Instrumentos de Escritura}
  \begin{itemize}
      \item Bolígrafos, lápices, y rotuladores de alta calidad.
      \item Marcadores fluorescentes para resaltar información importante.
      \item Borradores, correctores y otros elementos para edición.
  \end{itemize}
  
  \item \textbf{Materiales para Prototipado Rápido}
  \begin{itemize}
      \item Maquetas y kits de modelado para crear representaciones físicas de productos.
      \item Herramientas de corte como tijeras, cuchillas y cortadores de precisión.
  \end{itemize}
  
  \item \textbf{Equipos de Comunicación}
  \begin{itemize}
      \item Teléfonos y tablets para comunicaciones internas y externas.
      \item Software y aplicaciones de gestión de proyectos (como Trello o Asana).
      \item Cámaras y micrófonos para grabaciones de pitch y presentaciones.
  \end{itemize}
  
  \item \textbf{Materiales de Marketing y Publicidad}
  \begin{itemize}
      \item Papel especial y tintas para impresión de calidad de folletos y tarjetas de presentación.
      \item Materiales para merchandising como camisetas, gorras, y otros artículos personalizables.
      \item Sellos y material para embalaje y envío de productos.
  \end{itemize}
  
  \item \textbf{Recursos Digitales y Tecnológicos}
  \begin{itemize}
      \item Acceso a plataformas de aprendizaje en línea y recursos educativos.
      \item Software de diseño gráfico y edición (como Adobe Suite).
      \item Impresoras de alta resolución para material promocional y documentación.
  \end{itemize}
\end{itemize}